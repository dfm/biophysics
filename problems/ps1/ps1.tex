\documentclass[11pt]{article}
\usepackage{fullpage}
\usepackage{fancyhdr}

\usepackage{amsmath}
\usepackage{amssymb}

\usepackage{listings}
\lstset{language=Python,
        basicstyle=\footnotesize\ttfamily,
        showspaces=false,
        showstringspaces=false,
        tabsize=2,
        breaklines=false,
        breakatwhitespace=true,
        identifierstyle=\ttfamily,
        keywordstyle=\bfseries,
        commentstyle=\it,
        stringstyle=\it,
    }

\usepackage[pdftex]{graphicx}

% header
\fancyhead{}
\fancyfoot{}
\fancyfoot[C]{\thepage}
\fancyhead[R]{Daniel Foreman-Mackey}
\fancyhead[L]{Biophysics --- Problem Set 1}
\pagestyle{fancy}
\setlength{\headsep}{25pt}

% shortcuts
\newcommand{\Eq}[1]{Equation (\ref{eq:#1})}
\newcommand{\eq}[1]{Equation (\ref{eq:#1})}
\newcommand{\eqlabel}[1]{\label{eq:#1}}
\newcommand{\Fig}[1]{Figure \ref{fig:#1}}
\newcommand{\fig}[1]{Figure \ref{fig:#1}}
\newcommand{\figlabel}[1]{\label{fig:#1}}

% commands
\newcommand{\bvec}[1]{\ensuremath{\boldsymbol{#1}}}
\newcommand{\unit}[1]{\ensuremath{\,\mathrm{#1}}}

\begin{document}

% === Problem 1 ===
\section{Problem 1 --- How much DNA?}

According to Wolfram Alpha, the space between two base pairs in a chain of DNA
is 0.34nm and each base pair weighs in at about 660 daltons (or
$\sim1\times10^{-24}\unit{kg}$) \cite{bp-wa}.
To estimate the genome sizes, I downloaded the genome reports from NCBI
\cite{genome-reports} and computed the mean genome size listed in
\texttt{viruses.txt}, \texttt{prokaryotes.txt}, and \texttt{eukaryotes.txt}.

\paragraph{(a)}
From the NCBI report, the average virus genome size is about 40kbp.
Using the number from above and the estimated $\sim10^{30}$ virus cells, the
total length of viral DNA is $\sim10^{25}\unit{m} \approx
1.4\times10^9\unit{ly}$.
This DNA would weigh about $4\times10^{10}\unit{kg}$.

\paragraph{(b)}
The NCBI report gives an average prokaryote genome size of 4Mbp.
Applying the same procedure to this genome, I find the set of all prokaryote
DNA would stretch $\sim10^{27}\unit{m} \approx 1.4\times10^{11}\unit{ly}$ and
weigh $4\times10^{12}\unit{kg}$.

\paragraph{(c)}

According to Wolfram Alpha (and Wikipedia), the size of the human genome in
the average cell is about 3Gbp \cite{human-genome-wa}.
This corresponds to about a meter of DNA per cell!
Therefore, if a human has $10^{13}$ cells and if we say that there are
$6\times10^9$ people in the world, there will be $6\times10^{22}\unit{m}
\approx 6.5\times10^6\unit{ly}$ worth of human DNA weighing
$2\times10^8\unit{kg}$.


% === Problem 2 ===
\section{Problem 2 --- Fraction of Tubulin}

\paragraph{(a)}
Using a print out of this figure and a ruler, I estimated the total length of
microtubules in cell as $\sim28\unit{\mu m}$.
The density of tubulin is $1690 \unit{tubulin\,dimers\,\mu m^{-1}}$
\cite{mtiv,amos}.
This suggests that this cell contains about 47,320 tubulin dimers or 94,640
molecules.

\paragraph{(b)}
To estimate the volume, I'll model this cell as a cylinder where the depth
(out of the slide) is equal to the shorter edge.
Under this model, the total volume of the cell is
\begin{eqnarray}
\pi \, (2.2\unit{\mu m})^2 \, 6.1\unit{\mu m} &=& 9.3\times 10^{-17} \unit{m^3}
\quad.
\end{eqnarray}
Assuming a concentration of $10\unit{\mu M} = 10^{-2}\unit{mol\,m^{-3}}$, I
find that there are $9.3\times10^{-19}\unit{mol}$ of tubulin or $\sim56,000$
molecules.

\paragraph{(c)}
Combining the previous two results, I find that about 63\% of the tubulin is
polymerized in microtubules and about 37\% is in solution.


% === Problem 3 ===
\section{Problem 3 --- Size and composition of organelles}


% === Problem 4 ===
\section{Problem 4 --- Phase microscopy}


% === Problem 5 ===
\section{Problem 5 --- Adaptive significance}

\cite{gerhart, alon, lynch}

% === Bibliography ===
\bibliography{ps1}{}
\bibliographystyle{plain}

\end{document}
