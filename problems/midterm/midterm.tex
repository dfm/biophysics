\documentclass[12pt]{article}
\usepackage{fullpage}
\usepackage{fancyhdr}

\usepackage{amsmath}
\usepackage{amssymb}
\usepackage{url}
\usepackage[multiple]{footmisc}

\usepackage{listings}
\lstset{language=Python,
        basicstyle=\footnotesize\ttfamily,
        showspaces=false,
        showstringspaces=false,
        tabsize=2,
        breaklines=false,
        breakatwhitespace=true,
        identifierstyle=\ttfamily,
        keywordstyle=\bfseries,
        commentstyle=\it,
        stringstyle=\it,
    }

\usepackage[pdftex]{graphicx}

% header
\fancyhead{}
\fancyfoot{}
\fancyfoot[C]{\thepage}
\fancyhead[R]{Daniel Foreman-Mackey}
\fancyhead[L]{Biophysics --- Midterm}
\pagestyle{fancy}
\setlength{\headsep}{25pt}
\setlength{\headheight}{14.5pt}

% shortcuts
\newcommand{\Eq}[1]{Equation (\ref{eq:#1})}
\newcommand{\eq}[1]{Equation (\ref{eq:#1})}
\newcommand{\eqlabel}[1]{\label{eq:#1}}
\newcommand{\Fig}[1]{Figure \ref{fig:#1}}
\newcommand{\fig}[1]{Figure \ref{fig:#1}}
\newcommand{\figlabel}[1]{\label{fig:#1}}

\newcommand{\etal}{\emph{et al.}}

% commands
\newcommand{\bvec}[1]{\ensuremath{\boldsymbol{#1}}}
\newcommand{\unit}[1]{\ensuremath{\,\mathrm{#1}}}
\def\baselinestretch{1.2}

\begin{document}

\section*{Part 1}

Instead of researching the spontaneous organization of biological molecules
like I had planned\footnote{I did however look into the research by N. Clark
\emph{et al.} and it's very interesting, thanks!}, after we discussed
it in class, I decided to look into {\bf the structure of chromatin}.
In particular, I was especially interested in the techniques behind 3C so I'll
focus on that.
The other sub-area that I consider is FISH studies of ``chromosome
territories''.

\subsection*{Part 1.3}

\subsubsection*{Chromosome territories}

\emph{Reviews: \cite{ct-review3}, \cite{ct-review2} and \cite{ct-review1}}

\noindent\emph{Key papers: \cite{hamster}, \cite{cremer}, and \cite{local}}

It was long speculated---starting as early as the late 1800s
\cite{old}---that chromosomes are spatially isolated---in regions now called
``chromosome territories'', even during interphase \cite{ct-review1,
ct-review2}.
This theory was at odds with the competing---and then more popular
theory---that chromatin were arranged as a fully-mixed equilibrium
``spaghetti'' in the nucleus and it was mostly disregarded until the 1970s
\cite{cremer}.
The first convincing evidence of the territorial chromosome theory appeared
when members of the Cremer lab in Heidelberg, Germany determined that only a
small number of chromosomes were affected by focused radiation damage applied
to the DNA during interphase \cite{zorn, hamster}.
In the 80s and 90s, this theory was confirmed and more directly studied using
fluorescence \emph{in-situ} hybridization (FISH) \cite{cremer}.
This technique lets researchers tag and image the chromatin of specific
chromosomes, by targeting and linking fluorescent probes to specific DNA
sequences.
This image can then be used to directly see the location of each chromosome
even during interphase.
The fact that chromosomes are localized during interphase certainly has
epigenetic effects and recently, there has been research aimed at determining
and understanding the locations of specific chromosomes \cite{local}.

Much of the research in this field came from brothers {\bf Thomas Cremer}
(University of Munich, Germany) and {\bf Christoph Cremer} (University of
Heidelberg, Germany).
These researchers presented the first compelling evidence for the chromosome
territory theory \cite{hamster}.
In this paper, they subjected cells in interphase to focused UV damage and
found that if they followed the cells through to mitosis, the damage was only
visible on a few specific chromosomes.
In contrast, if the chromosomes get distributed and mixed during interphase
then you would expect the damage to be apparent on all the chromosomes.
This paper has 205 citations%
\footnote{\url{http://scholar.google.com/scholar?cites=1373788124891469070}}.


The Cremer brothers went on to use the FISH method for imaging the cell during
interphase \cite{cremer}.
By carefully tagging specific DNA sequences from different chromosomes with
different colors, this research confirmed the existence of chromosome
territories.
This paper has 490 citations%
\footnote{\url{http://scholar.google.com/scholar?cites=2575496365450001910}}.

Another important researcher in this field is {\bf Wendy Bickmore} (University
of Edinburgh).
A key paper out of her lab \cite{local} demonstrated that the locations of
specific chromosomes in human cells are set early and reproducibly.
In particular, they showed that two similar (in terms of size and the amount
of genetic information content) human chromosomes (18 and 19) occupy very
different places from each other; this distinction is observed in all cells.
This motivates the epigenetic theory that the layout of chromosomes in the
nucleus is actually functional and not random.
This paper has 654 citations%
\footnote{\url{http://scholar.google.com/scholar?cites=4847129722710273741}}.


\subsubsection*{Chromosome conformation capture}

\emph{Reviews: \cite{3c-review1}, \cite{3c-review2}, and \cite{3c-review3}}

\noindent\emph{Key papers: \cite{ccc}, \cite{looping}, and the set of
generalizations described in \cite{4c}, \cite{5c}, and \cite{hi-c}}

Chromosome conformation capture (3C) is an impressive statistical technique
for determining the distribution of chromatin in the nucleus.
The basic idea is to cross-link chromatin filaments that physically touch and
then sequence the cross-linked strands to determine which parts of the genome
are co-spatial \cite{3c-review1}.
The method is statistical because, in general, you must study a large set of
cells so that random interactions don't dominate the signal \cite{3c-review2}.
The formal steps in the procedure are \cite{3c-review1, 3c-review2}:
\begin{enumerate}

\item Fix the cells with formaldehyde, covalently linking interacting strands
of chromatin,

\item Digest the chromatin with a specific ``restriction enzyme'' that
separates the cross-linked chromatin from the rest,

\item Ligate the cross-linked strands to join them and get a single strand,

\item Dissolve the cross-links and sequence the strands.

\end{enumerate}
After this procedure, the frequency with which sequences are paired is related
to their average physical separation in the nucleus.
The original version of this procedure was published in a seminal 2002 paper
\cite{ccc} where the authors measured the layout of the genome of yeast cells
at various stages of meiosis.
Using this statistical sample of sequence distances, these authors
reconstructed the ensemble-averaged, three-dimensional structure of one of the
chromosomes in the yeast they were studying.

The three key researchers in this field that I would nominate for the Nobel
are {\bf Job Dekker}, {\bf Nancy Kleckner}, and {\bf Wouter de Laat}.
Dekker, now a professor at the University of Massachusetts, developed the 3C
method while a Postdoctoral Fellow in the Kleckner lab at Harvard \cite{ccc}.
Their paper \cite{ccc} described the 3C procedure and, as described above,
they applied it to determine the three-dimensional structure of a chromosome
in yeast.
This is the seminal paper of this field and it has been cited about 1250
times%
\footnote{\url{http://scholar.google.com/scholar?cites=14717049323260909894}}.

The next key paper \cite{looping} is from the de Laat lab.
In this paper, the authors used 3C to study a physical mechanism for
transcription regulation.
In particular, they showed that in liver cells of the murine---where the
$\beta$-globin protein is expressed---the ``locus control region'' (LCR; the
genes responsible for regulating the expression of the $\beta$-globin gene)
comes into physical contact with the $\beta$-globin gene even though it is
located tens of kilobases away in the genome.
In contrast, in brain cells where the protein does not occur, there is no
physical contact between the $\beta$-globin gene and the LCR.
This paper has been cited about 865 times%
\footnote{\url{http://scholar.google.com/scholar?cites=15030508233271733256}}.
At the time of this publication, de Laat was based at Erasmus University in
Rotterdam, Netherlands but he is now a professor at The Hubrecht Institute in
Utrecht.

Both Dekker \cite{5c, hi-c} and de Laat \cite{4c} have also worked to develop
more robust versions of the procedure called 5C, Hi-C, and 4C respectively.
Each method is designed to improve the flexibility and resolution of the
method by focusing on many smaller fragments of chromatin.
The papers have 412, 1232, and 579 citations%
\footnote{\url{http://scholar.google.com/scholar?cites=11459042540325628167}}%
\footnote{\url{http://scholar.google.com/scholar?cites=12390817964819514773}}%
\footnote{\url{http://scholar.google.com/scholar?cites=17950290286653863144}}
respectively.

\subsection*{Part 1.4}

{\bf I chose to focus on 3C for my review} mostly because I found it most
interesting technically.
That being said, it is also an exciting young field with a lot of recent
results that seem, from my research, to have revolutionized the way we study
the structure of chromatin.

\newpage

\section*{Part 2}

It has been known for well over a century that the genetic information in the
nuclei of eukaryotic cells organize into structured, distinct components
during cell division.
Shortly after this discovery, it was also postulated that even during
interphase the chromatin did not exist in equilibrium and instead, each
chromosome occupies a physically separate ``territory'' \cite{old}.
There was little evidence supporting this hypothesis and it fell mostly out of
favor until the 1970s when experiments---mostly from the Cremer lab in
Heidelberg, Germany---began to demonstrate that, indeed, this theory was
correct \cite{zorn, hamster, cremer, local}.
Much of this progress was due to technological advances; in particular, the
fluorescence \emph{in-situ} hybridization (FISH) method enabled specific
chromosomes to be fluorescently labeled with different colors and imaged
simultaneously in the same nucleus, providing the smoking gun for the
``chromosome territory'' theory.

After this discovery, the methodological development didn't slow.
In 2002, Job Dekker, Nancy Kleckner, and collaborators published their
\emph{chromosome conformation capture} (3C) method \cite{ccc}.
Unlike FISH, this technique provides precise, high-resolution, information
about the spatial structure of chromatin during interphase.
3C is a statistical technique for determining the typical physical separation
between specific sequences in a genome at any stage of the cell cycle.
The procedure is, roughly, \emph{(a)} fix a culture of cells with formaldehyde
to cross-link strands of DNA that are in physical contact, \emph{(b)} digest
the chromatin with a specific restriction enzyme, \emph{(c)} ligate the
digested fragments to form single strands, and \emph{(d)} sequence the
resulting fragments \cite{3c-review1}.
This dataset can then be used to determine the frequency at which specific
sequences come into contact and this can, in turn, be used to statistically
reconstruct the three-dimensional structure of a chromosome or the genome as a
whole \cite{ccc}.

In the original paper describing the 3C method \cite{ccc}, Dekker \etal\
applied the procedure to a specific chromosome yeast.
By studying the cells at various stages of meiosis, these authors demonstrated
quantitative measurements of the qualitative observations of chromosome
reorganization during cell division.
Then they studied the frequency of pairwise interactions between a set of
positions on the same chromosome to build a statistical sketch of the full
three-dimensional structure of the chromosome during interphase.

Shortly after this procedure was published, Wouter de Laat and members of his
lab in the Netherlands used it to study the physical interactions between
sequences describing specific proteins and their regulators in mammalian
cells \cite{looping}.
In this work, these authors found that the chromatin folds or ``loops'' to
place these sequences in physical contact in liver cells where the
$\beta$-globin protein is expressed but not in brain cells where the protein
is not found.
This was the first direct measurement of an epigenetic or physical basis for
the regulation of gene expression.

More recently, the focus of experiments like these has shifted to larger scale
measurements mapping entire genomes.
This type of analysis was not possible with the original 3C procedure because
the target sequences where the cross-linking frequency would be measured
needed to be chosen in advance and the specific reagents used were problem
specific.
These constraints have been relaxed by a few generalizations of the 3C method
\cite{4c, 5c, hi-c}.
Of these, the recent \emph{Hi-C} method \cite{hi-c} from the Dekker lab is the
most exciting.
Hi-C uses high-throughput sequencing to generate unbiased genome-wide
measurements of physical interactions between genetic loci.
For example, the Hi-C method was first applied to study the organization of
chromatin in human cells \cite{hi-c}.
In this study, the authors confirmed, quantitatively, previous qualitative
observations---made using FISH---that some chromosomes (16, 17, 19, 20, 21,
and 22) are concentrated near the interior of the nucleus while others are
preferentially located near the membrane (chromosome 18 in particular).
They also demonstrate that their data prefer the ``fractal model'' of
chromatin organization---where at every resolvable scale there exist
self-interacting loci---instead of an equilibrium model.

The development of 3C-type methods has revolutionized our understanding of the
organization of chromatin in the cell by enabling large-scale quantitative
studies of the physical interactions between genetic loci.
These methods are key if we hope to study the epigenetic effects of the
spatial distribution of chromatin.
The main shortcoming of these methods is that they are all population-level
experiments; they measure the ensemble-averaged properties of the chromosome
organization.
There has been some recent work on modifying these procedures to work with
single cells or, at least, study the variance in individual systems away from
the population averages, for example \cite{single, single2}, but there is
still work to be done!


\newpage

% === Bibliography ===
\bibliography{midterm}{}
\bibliographystyle{unsrt}

\end{document}
