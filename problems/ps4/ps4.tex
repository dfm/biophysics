\documentclass[12pt]{article}
\usepackage{fullpage}
\usepackage{fancyhdr}

\usepackage{amsmath}
\usepackage{amssymb}

\usepackage{listings}
\lstset{language=Python,
        basicstyle=\footnotesize\ttfamily,
        showspaces=false,
        showstringspaces=false,
        tabsize=2,
        breaklines=false,
        breakatwhitespace=true,
        identifierstyle=\ttfamily,
        keywordstyle=\bfseries,
        commentstyle=\it,
        stringstyle=\it,
    }

\usepackage[pdftex]{graphicx}

% header
\fancyhead{}
\fancyfoot{}
\fancyfoot[C]{\thepage}
\fancyhead[R]{Daniel Foreman-Mackey}
\fancyhead[L]{Biophysics --- Problem Set 4}
\pagestyle{fancy}
\setlength{\headsep}{25pt}

% shortcuts
\newcommand{\Eq}[1]{Equation (\ref{eq:#1})}
\newcommand{\eq}[1]{Equation (\ref{eq:#1})}
\newcommand{\eqlabel}[1]{\label{eq:#1}}
\newcommand{\Fig}[1]{Figure \ref{fig:#1}}
\newcommand{\fig}[1]{Figure \ref{fig:#1}}
\newcommand{\figlabel}[1]{\label{fig:#1}}

% commands
\newcommand{\bvec}[1]{\ensuremath{\boldsymbol{#1}}}
\newcommand{\unit}[1]{\ensuremath{\,\mathrm{#1}}}

\begin{document}

% === Problem 1 ===
\section{Problem 1}

\paragraph{(a)}

See \fig{curve}.


\paragraph{(b)}

The Hessian of this surface is
\begin{eqnarray}
H &=& \left( \begin{array}{cc}
    2 & 1 \\
    1 & -4
\end{array}\right) \quad.
\end{eqnarray}
We can find the eigenvalues $k_\pm$ of this matrix by solving
\begin{eqnarray}
{k_\pm}^2 + 2\,k_\pm - 9 &=& 0
\end{eqnarray}
giving
\begin{eqnarray}
k_\pm &=& \frac{-2 \pm \sqrt{40}}{2} \\
&\approx& 2.16 \quad \mathrm{and} \quad  -4.16 \quad.
\end{eqnarray}
These are the principal curvatures of the surface.
The mean curvature is therefore $H = -1$ and the Gaussian curvature is $K
= -9$.


\paragraph{(c)}

The bending free energy density is
\begin{eqnarray}
f &=& 2\,\kappa\,H + \kappa_G\,K
\end{eqnarray}
where $\kappa$ and $\kappa_G$ are the bending rigidities of the surface.
Since $H$ and $K$ are constant in $x$ and $y$, we only need to multiply by the
area of the square ($1$) to find the total free energy
\begin{eqnarray}
F &=& -2\,\kappa - 9 \kappa_G \quad.
\end{eqnarray}


\begin{figure}[htbp]
\begin{center}
\includegraphics[width=0.8\textwidth]{figure.pdf}
\end{center}
\caption{%
The surface $h(x,\,y)$ plotted as a function of $x$ and $y$ at a set of
different values of the other parameter.
\figlabel{curve}}
\end{figure}


% === Problem 2 ===
\section{Problem 2}


% === Problem 3 ===
\section{Problem 3}


% === Problem 4 ===
\section{Problem 4}


% === Problem 5 ===
\section{Problem 5}

This year's Nobel Prize in Physiology or Medicine was awarded to John O'Keefe,
May-Britt Moser, and Edvard I. Moser for their work on understanding how the
brain maps physical spaces \cite{press,nyt}.
In the early--mid 1970s, O'Keefe found that a specific combination of nerve
cells---he called them ``place cells''---in the hippocampus activated when
rats where in specific locations in their environment \cite{nobel1a, nobel1b}.
This suggests that these rats actually...
Building on this work, members of the Moser lab measured neurons in the
entorhinal cortex finding ``grid cells'' that seem to define a coordinate
system.



% === Bibliography ===
\bibliography{ps4}{}
\bibliographystyle{unsrt}

\end{document}
