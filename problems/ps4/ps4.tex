\documentclass[12pt]{article}
\usepackage{fullpage}
\usepackage{fancyhdr}

\usepackage{amsmath}
\usepackage{amssymb}

\usepackage{listings}
\lstset{language=Python,
        basicstyle=\footnotesize\ttfamily,
        showspaces=false,
        showstringspaces=false,
        tabsize=2,
        breaklines=false,
        breakatwhitespace=true,
        identifierstyle=\ttfamily,
        keywordstyle=\bfseries,
        commentstyle=\it,
        stringstyle=\it,
    }

\usepackage[pdftex]{graphicx}

% header
\fancyhead{}
\fancyfoot{}
\fancyfoot[C]{\thepage}
\fancyhead[R]{Daniel Foreman-Mackey}
\fancyhead[L]{Biophysics --- Problem Set 4}
\pagestyle{fancy}
\setlength{\headsep}{25pt}

% shortcuts
\newcommand{\Eq}[1]{Equation (\ref{eq:#1})}
\newcommand{\eq}[1]{Equation (\ref{eq:#1})}
\newcommand{\eqlabel}[1]{\label{eq:#1}}
\newcommand{\Fig}[1]{Figure \ref{fig:#1}}
\newcommand{\fig}[1]{Figure \ref{fig:#1}}
\newcommand{\figlabel}[1]{\label{fig:#1}}

% commands
\newcommand{\bvec}[1]{\ensuremath{\boldsymbol{#1}}}
\newcommand{\unit}[1]{\ensuremath{\,\mathrm{#1}}}

\begin{document}

% === Problem 1 ===
\section{Problem 1}

\paragraph{(a)}

See \fig{curve}.


\paragraph{(b)}

The Hessian of this surface is
\begin{eqnarray}
H &=& \left( \begin{array}{cc}
    2 & 1 \\
    1 & -4
\end{array}\right) \quad.
\end{eqnarray}
We can find the eigenvalues $k_\pm$ of this matrix by solving
\begin{eqnarray}
{k_\pm}^2 + 2\,k_\pm - 9 &=& 0
\end{eqnarray}
giving
\begin{eqnarray}
k_\pm &=& \frac{-2 \pm \sqrt{40}}{2} \\
&\approx& 2.16 \quad \mathrm{and} \quad  -4.16 \quad.
\end{eqnarray}
These are the principal curvatures of the surface.
The mean curvature is therefore $H = -1$ and the Gaussian curvature is $K
= -9$.

The principal radii of curvature are just
\begin{eqnarray}
r_\pm &=& \frac{1}{|k_\pm|} \\
&=& 0.24 \quad \mathrm{and} \quad 0.46 \quad.
\end{eqnarray}


\paragraph{(c)}

The bending free energy density is
\begin{eqnarray}
f &=& 2\,\kappa\,H + \kappa_G\,K
\end{eqnarray}
where $\kappa$ and $\kappa_G$ are the bending rigidities of the surface.
Since $H$ and $K$ are constant in $x$ and $y$, we only need to multiply by the
area of the square ($1$) to find the total free energy
\begin{eqnarray}
F &=& -2\,\kappa - 9 \kappa_G \quad.
\end{eqnarray}


\begin{figure}[htbp]
\begin{center}
\includegraphics[width=0.9\textwidth]{figure.pdf}
\end{center}
\caption{%
The surface $h(x,\,y)$ plotted as a function of $x$ and $y$ at a set of
different values of the other parameter.
\figlabel{curve}}
\end{figure}

\newpage


% === Problem 2 ===
\section{Problem 2}

I'm very sorry but I didn't leave myself enough time to finish this problem!
I was at a conference again.
I'll go over the result with my classmates after the fact and I'll try not to
let my travels get in the way of homework again.
I'm sorry!


% === Problem 3 ===
\section{Problem 3}

To form a spherical vesicle of radius $R$, the curvature of the surface must
be $k = 1/R$.
The Gaussian curvature is then $K = 1/R^2$ and the mean curvature is $H =
-1/R$ \cite{bending}.
This means that the energy required to bend a membrane into such a sphere is
\begin{eqnarray}
f &=& 2\,\kappa\,H + \kappa_G\,K \\
&=& \frac{2\,\kappa}{R} + \frac{\kappa_G}{R^2} \quad.
\end{eqnarray}


% === Problem 4 ===
\section{Problem 4}

\paragraph{(a)}

The classical ``fluid mosaic'' model of cell membranes \cite{fluid} describes
the membrane as mostly a lipid bilayer---exposed directly to the outside
world---with a few proteins floating unencumbered in this 2D surface.
Under this model, the membrane can be approximated as just a lipid bilayer
with small perturbations induced by the proteins.
Engelman \cite{membranes} argues that more recent experimental and theoretical
progress suggests that proteins actually heterogeneously fill much more of the
surface than this model would allow and that the membrane itself varies
substantially in structure.
Part of this effect comes from the fact that the proteins are all quite
varied (in shape and size) and they all interact with the membrane in
different complicated ways.
In this model the membrane structure is ``patchy'' instead of homogeneous.
One major result of a model like this is that much of the lipid bilayer that
was previously thought to be exposed to the aqueous environment is now thought
to be covered by large protruding protein complexes.
Another big change is that the interactions between proteins are now
non-negligible.
In the fluid mosaic model, proteins can be modeled as independent,
free-floating objects randomly distributed in the membrane surface.
In the ``patchy'' model, the proteins are not distributed randomly.
Instead, they are clustered---probably due to electrostatic interactions---in
specific functional groups that perform explicit tasks.

\paragraph{(b)}

For the second part of this problem, I read the paper by McMahon \& Gallop
\cite{curvature} describing some (primarily theoretical) mechanisms with which
cellular membranes can be deformed and vesicles can be created.
In particular, they describe five classes of mechanisms but emphasize that
the different types probably act in parallel.
These mechanisms can be summarized as follows:
\begin{enumerate}

{\item {\bf changing lipid composition}:
the two ways that lipid composition can effect the curvature of the membrane
are by \emph{(a)} chemically changing the headgroup size, or \emph{(b)} moving
individual lipids from one side of the membrane to the other.}

{\item {\bf integral membrane proteins}:
the shapes of transmembrane proteins are not known but if one has an
asymmetric (or, more specifically, conical) shape, the membrane will deform
around it and curve.}

{\item {\bf cytoskeleton}:
these authors make the argument that the cytoskeleton is \emph{probably}
involved in the deformation of membranes because it is involved in other
active changes in the cell---and, in particular, in response to membrane
tension---but there isn't really a clear description of what roll it exactly
plays.}

{\item {\bf peripheral membrane proteins}:
proteins can attach to the outside of the membrane and form a sort of
``exoskeleton'' to stabilize and sculpt the membrane shape.
Other ``coat'' proteins can indirectly influence the shape of the membrane by
forming curved structures that aren't directly attached to the membrane.}

{\item {\bf alpha helix insertion}:
the final method for inducing curvature that these authors discuss is when
small proteins---alpha helices---are inserted into one half of the lipid
bilayer causing the layer to become asymmetric.}

\end{enumerate}




% === Problem 5 ===
\section{Problem 5}

This year's Nobel Prize in Physiology or Medicine was awarded to John O'Keefe,
May-Britt Moser, and Edvard I. Moser for their work on understanding how the
brain maps physical spaces \cite{press,nyt}.
In the early--mid 1970s, O'Keefe found that a specific combination of nerve
cells---he called them ``place cells''---in the hippocampus activated when
rats where in specific locations in their environment \cite{nobel1a, nobel1b}.
This corresponds to measuring and remembering the context at each location.
These locations are then positioned absolutely in space by a grid of neurons
in the entorhinal cortex, discovered about thirty years later by members of
the Moser lab \cite{nobel2a, nobel2b, nobel2c}.
These grid cells are arranged in a uniform hexagonal grid in the brain and the
layout is mapped directly onto the physical space.

This work was awarded the Nobel Prize partly because it provides direct deep
insight into how the brain works but it also seems like it was also awarded
because there have been connections between degradation of these parts of the
brain and Alzheimer's disease.
There is therefore a chance that studying this neurological system will inform
research into treatments for Alzheimer's \cite{press}.


% === Bibliography ===
\bibliography{ps4}{}
\bibliographystyle{unsrt}

\end{document}
