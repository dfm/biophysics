\documentclass[12pt]{article}
\usepackage{fullpage}
\usepackage{fancyhdr}

\usepackage{amsmath}
\usepackage{amssymb}

\usepackage{listings}
\lstset{language=Python,
        basicstyle=\footnotesize\ttfamily,
        showspaces=false,
        showstringspaces=false,
        tabsize=2,
        breaklines=false,
        breakatwhitespace=true,
        identifierstyle=\ttfamily,
        keywordstyle=\bfseries,
        commentstyle=\it,
        stringstyle=\it,
    }

\usepackage[pdftex]{graphicx}

% header
\fancyhead{}
\fancyfoot{}
\fancyfoot[C]{\thepage}
\fancyhead[R]{Daniel Foreman-Mackey}
\fancyhead[L]{Biophysics --- Problem Set 2}
\pagestyle{fancy}
\setlength{\headsep}{25pt}

% shortcuts
\newcommand{\Eq}[1]{Equation (\ref{eq:#1})}
\newcommand{\eq}[1]{Equation (\ref{eq:#1})}
\newcommand{\eqlabel}[1]{\label{eq:#1}}
\newcommand{\Fig}[1]{Figure \ref{fig:#1}}
\newcommand{\fig}[1]{Figure \ref{fig:#1}}
\newcommand{\figlabel}[1]{\label{fig:#1}}

% commands
\newcommand{\bvec}[1]{\ensuremath{\boldsymbol{#1}}}
\newcommand{\unit}[1]{\ensuremath{\,\mathrm{#1}}}

\begin{document}

% === Problem 1 ===
\section{Problem 1 --- Einstein's Diffusion}

\paragraph{(a)}

In his 1905 paper \cite{einstein}, Einstein derived the mathematical
formalism for the stochastic Brownian motions of a particle suspended in a
liquid.
By considering a volume


\paragraph{(b)}




% === Problem 2 ===
\section{Problem 2 --- Life at Low Reynolds Number}

\paragraph{(a)}

Purcell's \emph{Life at Low Reynolds number} \cite{purcell} really is a
pleasure to read!
The main point of the paper is to explore some of the surprising and
unintuitive dynamics at low Reynolds number.
In particular, he spends most of the talk describing how small organisms---or
``bugs''---might propel themselves (``swim'') in the effective absence of
inertia.
The last section describes a possible motivation for these motions and this
again results in a discussion of the unintuitive physics in this limit.

The first unintuitive result starts with the scallop example described below.
The main argument here is that, since Navier-Stokes becomes time independent
at low Reynolds number, a scallop could not swim in such conditions.
Instead, the simplest ``creature'' that could swim at low Reynolds number
would need two degrees of freedom.
In the example given in Figure 7, I think that this animal would swim from
right-to-left (not left-to-right as he seems to imply in the text).

Purcell then moves on to discuss a theory of how a creature with a flagellum
might move.
Starting with experimental evidence showing that flagella are powered by some
sort of rotating motor, Purcell argues that the direction of motion of the
creature is not immediately obvious.
Instead of displacing the surrounding medium to move, the mechanism is more
similar to a corkscrew.
In the paper, Purcell considers a corkscrew in a viscous liquid but I think
that the thought experiment would also work if you consider opening a bottle
of wine.
In that case, you're again applying a torque to generate a translation.

The final theory in this talk concerns the interesting question of why these
microscopic creates would want to swim.
By making a few (probably reasonable) assumptions to simplify the problem,
Purcell argues that diffusion is still going to be the main mechanism for
these creatures to obtain nutrients (instead of stirring the medium to collect
them).
However, they could swim to regions with higher nutrient density by moving
faster than the rate of diffusion.
Purcell tentatively argues that the measurements in Figure 11 suggest that
this is correct (and it seems convincing) but I don't know if this is a
commonly held belief.


\paragraph{(b) The Scallop Theorem}

The first example of low Reynolds number dynamics is a scallop in a viscous
liquid.
Apparently, scallops swim by opening and quickly closing their valves, thus
displacing water and moving in the expected direction \cite{scallop}.
If a scallop were to find itself in a low Reynolds number situation, it would
not be able to move.
In the absence of inertia, the Navier-Stokes equation is no longer time
dependent and since the only motion that a scallop can perform is
``reciprocal''---the time-reversed configurations are identical to the
original---it can't generate any net displacement.
Instead, when closing its valves, the scallop simply reverses the motion
produced when opening them.
This restriction will apply to any organism with only one degree of freedom in
configuration space but although the motion of a fish's tail might appear
somewhat similar, I think that fish would still be able to swim at low
Reynolds number because the tail is flexible so it effectively has more
degrees of freedom; this is the ``flexible oar'' model in the paper.


% === Bibliography ===
\bibliography{ps2}{}
\bibliographystyle{unsrt}

\end{document}
